\chapter{Referencial Teórico}\label{cap:referencial}



\section{Fundamentos da computação quântica}



\subsection{Qubits}
% Aqui falaremos sobre qubits, a estrututa básica da computação quântica

O bit quântico (\textit{quantum bit}), ou simplesmente qubit, é a unidade fundamental da computação quântica, assim como o bit é para a computação clássica. Comparar um bit clássico com um qubit ajuda a compreender sua natureza. Um bit clássico pode assumir apenas um de dois valores possíveis: $0$ ou $1$. Enquanto nada o modifique, ele permanecerá em um desses dois estados fixos \cite{sutor2019dancing}. 

De maneira análoga, quando medimos um qubit, obtemos um de dois resultados possíveis: $\ket{0}$ ou $\ket{1}$ (a notação de \textit{bras} e \textit{kets} será abordada em capítulo posterior). A principal diferença é que, antes da medição, o qubit pode existir em um estado de superposição, descrito pela combinação linear:
\[
\ket{\psi} = \alpha\ket{0} + \beta\ket{1}, \quad \text{com } |\alpha|^2 + |\beta|^2 = 1,
\]
onde $\alpha$ e $\beta$ são amplitudes complexas associadas às probabilidades de o qubit ser encontrado em $\ket{0}$ ou $\ket{1}$, respectivamente \cite{sutor2019dancing}. 

A superposição é um princípio fundamental da mecânica quântica que estabelece que, antes da medição, um sistema pode existir simultaneamente em múltiplos estados possíveis. No ato da medição, ocorre o chamado \textit{colapso da função de onda}, fazendo o sistema assumir um dos estados definidos. No caso de um qubit, os dois estados possíveis formam uma base bidimensional, e a superposição pode ser visualizada geometricamente na Esfera de Bloch (Fig.~\ref{fig:esfera}), que representa todos os estados quânticos possíveis de um qubit.

\begin{figure}[h]
    \centering
    \includegraphics[width=0.3\linewidth]{./Figuras/Bloch_sphere.png}
    \caption{Representação de um qubit na Esfera de Bloch.}
    \label{fig:esfera}
\end{figure}

Um equívoco comum é atribuir à superposição, isoladamente, a origem do alto poder computacional dos computadores quânticos. Embora ela seja condição necessária, não é suficiente. A superposição permite o \textit{paralelismo quântico}, em que múltiplas combinações de estados podem ser processadas simultaneamente. No entanto, sem mecanismos de interferência controlada, as medições resultariam apenas em resultados aleatórios, não necessariamente úteis para resolver um problema. 

É por meio da interferência entre amplitudes quânticas que os algoritmos quânticos exploram as propriedades da superposição para eliminar resultados incorretos e reforçar os corretos. Os algoritmos de Shor e de Grover exemplificam esse princípio: ambos utilizam a superposição e a interferência de forma controlada para atingir ganhos de desempenho significativos em relação aos métodos clássicos.



\section{Conceitos complementares}

Essa seção se dedica a explicar conceitos que formam a base da computação quântica


\section{Brakets}

Na computação quântica é extremamente comum o uso de vetores e matriz para a representação dos estados quânticos. De forma geral podemos ter vetores em forma de linha e em forma de coluna, vamos considerar o vetor $\textbf{v} = v_1, v_2, ..., v_n$

\begin{bmatrix}
    v_1 & v_2 & ... & v_nx
\end{bmatrix}




