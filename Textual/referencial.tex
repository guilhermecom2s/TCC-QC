\chapter{Referencial Teórico}\label{cap:referencial}



\section{Fundamentos da computação quântica}

\subsection{Qubits}
% Aqui falaremos sobre qubits, a estrututa básica da computação quântica

O bit quântico (\textit{quantum bit}), ou apenas qubit, é a unidade fundamental da computação quântica, assim como o bit é para a computação clássica. Compara um qubit clássico com um qubit pode nos ajudar a entender esse último. Um bit clássico terá sempre apenas um de dois valores possíveis: $0$ ou $1$ e, desde que nada aconteça para mudar o estado, o bit permanecerá com um desses dados valores \cite{sutor2019dancing}. De maneira análoga, quanto medimos um qubit, também pode-se obter um entre dois estados: $\ket{0}$ ou $\ket{1}$ (mais sobre essa notação no capítulo (incluir capítulo para falar sobre os brakets...)), porém, a grande diferença do qubit, é que antes de ser medido, ele pode estar em um estado de superposição, possuindo valores intermediários entre $\ket{0}$ e $\ket{1}$. Os estados de um qubit podem ser representados através da esfera de Bloch...






