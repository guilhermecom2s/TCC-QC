%==============================================================
%------------------JONATHAN TOBIAS DA SILVA--------------------
%--------------------ENGENHARIA ELÉTRICA-----------------------
%------------INSTITUTO FEDERAL- CAMPUS SERTÃOZINHO-------------
%-------------packages.tex, v1.0.0, jonathanTSilva-------------
%
%------------------github.com/JonathanTSilva-------------------
%---------------linkedin.com/in/JonathanTSilva-----------------
%
% Copyright, 2012–2021 by abnTeX2 at https://www.abntex.net.br/
%==============================================================

%--------------------LÍNGUA E CARACTERES:---------------------|
\usepackage[utf8]{inputenc}                                     % Codificação do documento (conversão automática dos acentos)
\usepackage[T1]{fontenc}                                        % Seleção de códigos de fonte
%\usepackage{fontspec}                                          % Seleção de fontes
%-------------------------------------------------------------|

%-----------------------MODO MATEMÁTICO-----------------------|
\usepackage{amsmath, amsfonts, amssymb}                         % Pacotes da sociedade americana de matemática ASM
%-------------------------------------------------------------|

%--------------------------CRIAÇÕES---------------------------|
%\usepackage{tikz}
%-------------------------------------------------------------|

%---------------------EXIBIÇÃO E FORMATAÇÃO-------------------|
\usepackage{graphicx}                                           % Inclusão de gráficos
\usepackage[table, svgname]{xcolor}                             % Controle de Cores
\usepackage{colortbl}
\usepackage{indentfirst}                                        % Indenta o primeiro parágrafo de cada seção
\usepackage{float}                                              % Fixar imagem no ponto específicado [H]
\usepackage{multirow, multicol}                                 % Tabelas com células multicolunas e multilinhas
\usepackage{enumitem}                                           % Controle sobre o layout dos três ambientes básicos de lista: enumerar, itemizar e descrição
\usepackage{microtype}                                          % Para melhorias de justificação
\usepackage{listings}                                           % Inserir código de linguagem de programação
\usepackage{nomencl}                                            % Necessário para o commando makeindex
%\usepackage{floatrow}
\usepackage{subfig}
\usepackage[size=footnotesize, skip=3pt]{caption}
%\usepackage{subcaption}
\usepackage{longtable}                                          % Para as tabelas
\usepackage{booktabs}
\usepackage{tabularx}
%-------------------------------------------------------------|

%---------------BIBLIOGRAFIA, LINKS E CITAÇÕES----------------|
\usepackage[brazilian, hyperpageref]{backref}	                % Paginas com as citações na bibl
\usepackage{xurl}
\usepackage{hyperref}                                           % Três pacotes necessários para a quebra de url longa
\usepackage[hyphenbreaks]{breakurl}
\usepackage[alf,abnt-emphasize=bf]{abntex2cite}	                % Citações padrão ABNT, quando classe abnTeX2
%\citeoption{minhasopcoes}                                      % Le as opções estabelecidas no arquivo .bib para abnTeX2cite
%-------------------------------------------------------------|

%---------------------------OUTROS----------------------------|
\usepackage{pdfpages}                                           % Inclui página de outro pdf. 
                                                                % Ex: \includepdf[pages={1,3-},nup=1x2,landscape=true]{main.pdf}
\usepackage{lastpage}                                           % Usado pela Ficha catalográfica
%-------------------------------------------------------------|

%---------------------------FONTES----------------------------|
\usepackage{lmodern}                                            % Usa a fonte Latin Modern
%-------------------------------------------------------------|